\documentclass[11pt,a4paper]{article}
\usepackage[utf8]{inputenc}
\usepackage[T1]{fontenc}
\usepackage[italian]{babel}
\usepackage{geometry}
\usepackage{natbib}
\usepackage{url}
\usepackage{hyperref}

\geometry{margin=2.5cm}
\bibliographystyle{plainnat}

\title{Bibliografia sulla Stereofonia\\Articoli Tecnici e Scientifici}
\author{}
\date{\today}

\begin{document}

\maketitle

\begin{abstract}
La presente bibliografia raccoglie gli articoli tecnici e scientifici fondamentali per lo studio della stereofonia, dalle prime intuizioni sperimentali di Clément Ader nel 1881 fino agli sviluppi contemporanei dell'audio immersivo. La raccolta è organizzata tematicamente per evidenziare l'evoluzione concettuale e tecnologica di questa disciplina.
\end{abstract}

\section{Preistoria della Stereofonia (1876--1930)}

I primi esperimenti di trasmissione sonora spaziale nascono dall'intuizione che la percezione uditiva umana è intrinsecamente tridimensionale. Clément Ader con il suo Théâtrophone realizza il primo sistema di trasmissione binaurale commerciale, precedendo di mezzo secolo la formalizzazione teorica.

\nocite{bell1877}
\nocite{edison1877}
\nocite{dolbear1879}
\nocite{preece1879}
\nocite{ader1881}
\nocite{ader1881patent}
\nocite{du_moncel1882}
\nocite{tainter1886}
\nocite{berliner1888}
\nocite{bartok1889}
\nocite{hospitalier1889}
\nocite{puskas1893}
\nocite{hunnings1899}
\nocite{koenig1899}
\nocite{electrician1925}
\nocite{fletcher1928}
\nocite{steinberg1929}
\nocite{wente1930}

\section{Origini e Fondamenti Teorici (1931--1940)}

Il brevetto di Blumlein del 1931 segna l'inizio della stereofonia moderna, fornendo le basi teoriche per la registrazione e riproduzione su due canali. Gli studi contemporanei di Fletcher e Steinberg ai Bell Labs gettano le fondamenta psicoacustiche della disciplina.

\nocite{blumlein1931}
\nocite{fletcher1934}
\nocite{steinberg1934}

\section{Sviluppi Tecnici (1950--1970)}

L'era della commercializzazione vede la nascita dei primi sistemi stereofonici domestici e le prime riflessioni sulla compatibilità tra sistemi mono e stereo. Clark, Dutton e Vanderlyn sviluppano i protocolli per la registrazione su nastro magnetico.

\nocite{clark1957}
\nocite{klapholz1958}
\nocite{bauer1961}
\nocite{schroeder1961}
\nocite{cooper1960}

\section{Michael Gerzon e l'Ambisonics (1970--1990)}

Gerzon rivoluziona il concetto di riproduzione spaziale introducendo l'Ambisonics, sistema che permette la codifica completa di un campo sonoro tridimensionale. I suoi lavori sulla perifonìa aprono la strada all'audio surround contemporaneo.

\nocite{gerzon1973}
\nocite{gerzon1975}
\nocite{gerzon1977}
\nocite{gerzon1992}

\section{Teoria della Localizzazione e Psicoacustica}

La comprensione dei meccanismi percettivi alla base della localizzazione sonora, dai primi studi di Lord Rayleigh fino alle ricerche contemporanee di Blauert e Moore, fornisce le basi scientifiche per lo sviluppo di sistemi di riproduzione sempre più accurati.

\nocite{rayleigh1907}
\nocite{mills1958}
\nocite{blauert1969}
\nocite{blauert1983}
\nocite{moore1995}

\section{Tecniche Microfoniche e Registrazione}

L'evoluzione delle tecniche di ripresa stereofonica, dall'analisi teorica di Lipshitz alle applicazioni pratiche di Williams e Le Maitre, definisce i protocolli per la cattura fedele dell'immagine sonora spaziale.

\nocite{lipshitz1986}
\nocite{williams1987}
\nocite{joseph2007}

\section{Elaborazione Digitale e Standard}

L'avvento del digitale trasforma radicalmente la stereofonia, permettendo algoritmi complessi di elaborazione e la definizione di standard internazionali per la compatibilità tra sistemi.

\nocite{bosi1997}
\nocite{itu2012}

\section{Audio Binaurale e HRTF}

Lo sviluppo delle tecniche binaurali, basate sulla modellizzazione delle Head-Related Transfer Functions, permette la creazione di esperienze sonore tridimensionali attraverso cuffie convenzionali.

\nocite{wightman1989}
\nocite{begault1994}
\nocite{algazi2001}
\nocite{cheng2001}

\section{Sviluppi Contemporanei e Audio Immersivo}

Le ricerche più recenti si concentrano su sistemi di riproduzione sempre più sofisticati: Vector Base Amplitude Panning, Wave Field Synthesis e Ambisonics di ordine superiore rappresentano il futuro dell'audio spaziale.

\nocite{berkhout1993}
\nocite{pulkki1997}
\nocite{daniel2003}
\nocite{frank2013}
\nocite{zotter2019}

\bibliography{stereophony}

\end{document}